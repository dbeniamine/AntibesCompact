%!TEX encoding=UTF-8 Unicode
% -
 %   Copyright (C) 2015  Beniamine, David <David@Beniamine.net>
 %   Author: Beniamine, David <David@Beniamine.net>
 %   
 %   This program is free software: you can redistribute it and/or modify
 %   it under the terms of the GNU General Public License as published by
 %   the Free Software Foundation, either version 3 of the License, or
 %   (at your option) any later version.
 %   
 %   This program is distributed in the hope that it will be useful,
 %   but WITHOUT ANY WARRANTY; without even the implied warranty of
 %   MERCHANTABILITY or FITNESS FOR A PARTICULAR PURPOSE.  See the
 %   GNU General Public License for more details.
 %   
 %   You should have received a copy of the GNU General Public License
 %   along with this program.  If not, see <http://www.gnu.org/licenses/>.
 %%

\documentclass[xcolor={usenames,dvipsnames}]{beamer}

%=========================Language and encoding ==============================

\usepackage[utf8]{inputenc}
\usepackage[english]{babel}
\usepackage[T1]{fontenc}
% Fix size errors due to T1 in bbl file
\usepackage{fix-cm}
%=============================================================================

%========================= Todo notes  =======================================

\usepackage{todonotes}
\presetkeys{todonotes}{inline}{}

%=============================================================================

%========================= Figures ===========================================

\usepackage{graphicx} % support the \includegraphics command and options
\graphicspath{ {./img/} }
%\usepackage{tikz}
\usepackage{epstopdf}
%\usepackage{subcaption}

%=============================================================================

%=============================================================================

%========================= Hyperref ==========================================

\usepackage{hyperref}
\hypersetup{
    hyperindex=true, %ajoute des liens dans les index.
    colorlinks=false, %colorise les liens
    breaklinks=true, %permet le retour à la ligne dans les liens trop longs
    urlcolor= blue, %couleur des hyperliens
    linkcolor= black, %couleur des liens internes
bookmarksopen=true,}

%=============================================================================

%========================= Other useful includes =============================

\usepackage{ifthen}
\usepackage[absolute,overlay]{textpos} %to set some blocks position
%=============================================================================

%========================= Beamer theme =====================================

%Stuff for printable version
\mode<handout>{
    \usetheme{default}
    \setbeamercolor{background canvas}{bg=black!5}
    \pgfpagesuselayout{4 on 1}[a4paper,landscape,border shrink=2.5mm]
}

\usetheme{AntibesCompact}
%=============================================================================

%========================= Title frame  ======================================
\title[AntibesCompact]{Antibes Compact: A simple yet pretty beamer theme}
\author[Dbeniamine]{David Beniamine}
\institute[No inst]{No institutution for this work}
%=============================================================================

\begin{document}
%========================= Title and outlines ================================
\begin{frame}{}
    \titlepage
\end{frame}

\newboolean{sectiontoc}
\setboolean{sectiontoc}{true} % default to true

\AtBeginSection[]
{
    \ifthenelse{\boolean{sectiontoc}}{
        \begin{frame}<beamer>
            \frametitle{Outline}
            \tableofcontents[currentsection,currentsubsection]
        \end{frame}
    }
}
\AtBeginSubsection[]
{
    \ifthenelse{\boolean{sectiontoc}}{
        \begin{frame}<beamer>
            \frametitle{Outline}
            \tableofcontents[currentsection,currentsubsection]
        \end{frame}
    }
}

%=============================================================================

%========================= Real presentation =================================

\begin{frame}{Let's start !}
    \begin{block}{A minimal design}
        \begin{itemize}
            \item Keep the focus on the presentation
            \item Preserve as much space as possible for your slides
            \item Yet show all information relative to presentation / speaker
        \end{itemize}
    \end{block}
\end{frame}

\begin{frame}{Outline}
    \tableofcontents
\end{frame}

\section{A first section}
\begin{frame}{Tree}
    Did you see the flat tree that appears in the red bar ?
\end{frame}
\subsection{A subsection}
\begin{frame}{Two level tree}
    \begin{alertblock}{Flat tree}
       With the classic tree theme, you would have lost one line in this frame.
    \end{alertblock}
\end{frame}
\subsection{Another subsection}
\begin{frame}{Dummy slide}
    \begin{exampleblock}{Because why not ?}
        And also to show an example block
    \end{exampleblock}
\end{frame}
\section*{A hidden section}
\begin{frame}{No idea}
    \begin{block}{So much space}
        Yet no idea what to do with it, I should stop here \ldots
    \end{block}
\end{frame}
\newcounter{finalframe}
\setcounter{finalframe}{\value{framenumber}}
%Last numbered frame go here
%=============================================================================

%=============================================================================
%Uncomment next lines for uncounted backup slides & biblio
%\section*{Bibliography}
%\setboolean{sectiontoc}{false}
%
%\bibliographystyle{apalike}
%\bibliography{biblio}

%========================= Backup slides =====================================
%\newsection*{Hidden slides}
%put this line before each frame
%\setcounter{framenumber}{\value{finalframe}}

%=============================================================================
\end{document}




